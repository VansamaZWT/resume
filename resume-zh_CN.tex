% !TEX TS-program = xelatex
% !TEX encoding = UTF-8 Unicode
% !Mode:: "TeX:UTF-8"

\documentclass{resume}
\usepackage{zh_CN-Adobefonts_external} % Simplified Chinese Support using external fonts (./fonts/zh_CN-Adobe/)
% \usepackage{NotoSansSC_external}
% \usepackage{NotoSerifCJKsc_external}
% \usepackage{zh_CN-Adobefonts_internal} % Simplified Chinese Support using system fonts
\usepackage{linespacing_fix} % disable extra space before next section
\usepackage{cite}
\usepackage{enumitem}

\begin{document}
\pagenumbering{gobble} % suppress displaying page number

\name{千早爱音}

\basicInfo{
  \email{211220136@smail.nju.edu.cn} \textperiodcentered\ 
  \phone{(+86) 13984341840}}
 
\section{\faGraduationCap\  教育背景}
\datedsubsection{\textbf{南京大学}}{2021年9月 -- 至今}
\textit{本科在读}\ 计算机科学与技术系, 预计 2025 年 6 月毕业
\begin{itemize}[parsep=0.5ex]
  \item GPA: 4.34 / 5.0, 排名: 前28.6\%
  \item 核心课程: 计算机系统基础, 算法设计与分析, 操作系统, 计算机网络, 形式语言与自动机, 编译原理
  \item 选修课程: 程序设计语言的形式语义, 软件分析, 组合数学
  \item 担任过SICP课程(2021年秋季学期)和操作系统课程(2023年春季学期, 大班)的助教
\end{itemize}

\section{\faHeartO\ 获奖情况}
\datedline{人民奖学金三等奖}{2023年12月}
\datedline{人民奖学金三等奖}{2022年12月}

\section{\faUsers\ 项目经历}
\datedsubsection{\textbf{类UNIX操作系统}}{2023年2月 -- 2023年6月}
\role{C, Linux}{南京大学操作系统课程(2023年春季学期, 大班)课程实验}
\begin{itemize}[parsep=0.5ex,topsep=0pt]
  \item 项目地址:https://git.nju.edu.cn/oslab2023/oslab
  \item 基于i386架构, 支持在分页虚拟内存上加载执行ELF文件, 分时多进程, 实现了类UNIX文件系统, 能在QEMU上执行
\end{itemize}

\vspace{0.5ex}

\datedsubsection{\textbf{Java程序静态分析器}}{2023年9月 -- 2023年12月}
\role{Java}{软件分析课程实验}
\begin{itemize}[parsep=0.5ex,topsep=0pt]
  \item 项目地址:https://github.com/pascal-lab/Tai-e-assignments
  \item 基于Tai-e框架(教学版)实现
  \item 包括多种数据流分析, 指针分析和污点分析
\end{itemize}

\vspace{0.5ex}

\datedsubsection{\textbf{i386指令集模拟器}}{2022年9月 -- 2023年1月}
\role{C, Linux}{计算机系统基础课程实验}
\begin{itemize}[parsep=0.5ex,topsep=0pt]
  \item 项目地址:https://github.com/ics-nju-wl/icspa-public-guide
  \item 一个完备的i386指令集模拟器(NEMU), 以及一个运行在NEMU上的小型操作系统
  \item 成功在NEMU上运行了游戏"仙剑奇侠传"
\end{itemize}



% Reference Test
%\datedsubsection{\textbf{Paper Title\cite{zaharia2012resilient}}}{May. 2015}
%An xxx optimized for xxx\cite{verma2015large}
%\begin{itemize}
%  \item main contribution
%\end{itemize}

\section{\faCogs\ 技能}
% increase linespacing [parsep=0.5ex]
\begin{itemize}[parsep=0.5ex]
  \item 编程语言: C > Python > C++ > Java
  \item 外语: 英语(六级534分)
\end{itemize}

%\section{\faInfo\ 其他}
% increase linespacing [parsep=0.5ex]
%\begin{itemize}[parsep=0.5ex]
%  \item 技术博客: http://blog.yours.me
%  \item GitHub: https://github.com/username
%  \item 语言: 英语 - 熟练(TOEFL xxx)
%\end{itemize}

%% Reference
%\newpage
%\bibliographystyle{IEEETran}
%\bibliography{mycite}
\end{document}
